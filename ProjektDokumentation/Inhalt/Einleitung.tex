% !TEX root = ../Projektdokumentation.tex
\section{Einleitung}
\label{sec:Einleitung}


\subsection{Projektumfeld} 
\label{sec:Projektumfeld}
\begin{itemize}
	\item Die folgende Projektdokumentation schildert den Ablauf des IHK-Abschlussprojektes, welches der Autor im Rahmen seiner Ausbildung zum Fachinformatiker mit Fachrichtung Anwendungsentwicklung
	durchgef�hrthat. Ausbildungsbetrieb ist die Fanuc Europe GmbH.
\end{itemize}

\subsection{Projektziel} 
\label{sec:Projektziel}
\begin{itemize}
	\item Im CRM-System der Fanuc Europe befinden sich �ber 220.000 Kundenkonten, die Daten wie
	Adresse, Telefonnummer und Website des Kunden enthalten. Zur Automatisierung der
	Verwaltung dieser Kunden wird die Salesforce CRM-Plattform verwendet.
	\item Da es unm�glich ist, �nderungen an den Kundendaten manuell zu verfolgen, ist die Erstellung
	eines Web-Scraping-Systems geplannt. Diese Anwendung automatisiert die jeweiligen
	Kontaktseiten Im Internet herauslesen und vergleicht sie mit den aktuellen Daten im
	CRM-System.
\end{itemize}


\subsection{Projektbegr�ndung} 
\label{sec:Projektbegruendung}
\begin{itemize}
	\item Das Unternehmen verf�gt oft �ber Informationen, die in der gesamten Organisation verstreut sind, d.h. teilweise redundant in getrennten Systemen, aber h�ufig unvollst�ndig.
	Das hat zur Folge, dass mehrere Unternehmensbereiche scheinbar nur mit den gleichen Daten arbeiten. 
	Die kritische 360-Grad-Perspektive von Kunden oder Gesch�ftspartnern fehlt somit. 
	Dar�ber hinaus besteht ein erhebliches Risiko, dass unzureichende Daten als Grundlage f�r strategische Gesch�ftsentscheidungen verwendet werden, die dann zur Festlegung von Unternehmenszielen und zur Modellierung von Gesch�ftsprozessen herangezogen werden. 
	Folglich sind qualitativ hochwertige Daten erforderlich, damit das Unternehmen auf dem Markt erfolgreich sein kann. Der Hauptzweck besteht darin, die manuelle Arbeit zu verringern, die h�ufig dazu f�hrt, dass ungenaue Daten im System gespeichert werden. 
    Dies f�hrt zu Problemen bei der Datenqualit�t im Unternehmen. 
	Stattdessen kann ein Administrator einmal pro Woche einen automatisierten Vorgang ausf�hren, um �ber �nderungen an den Kundendaten informiert zu werden, was die Produktivit�t erh�ht und den Zeitaufwand f�r die Suche nach den relevanten Informationen verringert.
	Aus den oben genannten Gr�nden beauftragte mich die Leitung der IT-Abteilung im Rahmen meines Abschlussprojekts mit der Verbesserung der Datenqualit�t durch eine C\#-Anwendung.
\end{itemize}

\subsection{Projektschnittstellen} 
\label{sec:Projektschnittstellen}
\begin{itemize}
	\item Das Hauptaugenmerk liegt auf der Bestimmung von Softwarekomponenten, die sich am effektivsten in die Betriebs- und Systemumgebung des Unternehmens integrieren lassen.
	In jedem Fall m�ssen bei der Auswahl dieser Softwarekomponenten die aktuellen CRM-, ERP- oder sonstigen datenhaltenden Systeme ber�cksichtigt werden. 
	Die aktuellen Daten werden �ber WSDL-Methoden von Salesforce in eine C\#-Anwendung transportiert, die WPF f�r die Pr�sentationsschicht nutzt und einem definierten MVVM-Design folgt. 
	Damit wird eine logische Schnittstelle geschaffen, �ber die alle Subsysteme agieren k�nnen. Ziel ist es, die Datenverarbeitung und Web Scraping Services so einfach wie m�glich zu gestalten.

\end{itemize}


\subsection{Projektabgrenzung} 
\label{sec:Projektabgrenzung}
\begin{itemize}
	\item Eine der gr��ten Herausforderungen bei der Erstellung dieses Projekts ist die Definition dessen, was geschehen soll.
	Daher ist es wichtig, den vom CRM-Entwicklungsteam ben�tigten Output klar zu definieren. 
	Es wurde klargestellt, dass ein System, das Daten innerhalb der CRM-Plattform automatisch �berschreibt, nicht erw�nscht ist, sondern den Benutzer mit Hilfe von logisch aufgebauten Berichten benachrichtigt.
	Dies f�hrt zu einer geringeren Fehleranf�lligkeit, da Websites dynamisch sein oder sich �ndernde Elemente aufweisen k�nnen, mit denen der Web Scraper Probleme bekommen kann.
\end{itemize}
