% !TEX root = ../Projektdokumentation.tex
\section{Projektplanung} 
\label{sec:Projektplanung}
\label In der Projektplanung soll die notwendige Zeit und die ben�tigten Ressourcen sowie ein Ablauf der
Durchf�hrung des Projektes geplant werden.


\subsection{Projektphasen}
\label{sec:Projektphasen}

\begin{itemize}
	\item F�r die Umsetzung des Projekts standen dem Autor 70 Stunden zur Verf�gung. Diese wurden vor dem Start des Projektes auf verschiedene Phasen verteilt, die w�hrend der Softwareentwicklung durchlaufen werden.
	Eine grobe Zeitplanung sowie die Hauptphasen k�nnen der Tabelle 1 entnommen werden: Grobzeitplanung
	entnommen werden. Dazu lassen sich die einzelnen Hauptphasen noch in kleinere Unterphasen untergliedern.
\end{itemize}

%\tabelle{ProjektplanKurz}{tab:Projektplan}{ProjektplanKurz.tex}

\subsection{Abweichungen vom Projektantrag}
\label{sec:AbweichungenProjektantrag}

\begin{itemize}
	\item Im Projektantrag wurde festgelegt, dass bestimmte Bibliotheken mit Python f�r das Webscraping-System zusammen mit einem einfachen Codebeispiel verwendet werden.
	 Dies wurde jedoch ge�ndert, da es mehrere dynamische Websites gibt, was bedeutet, dass der Benutzer zuerst mit der Website interagieren muss, um auf sie zuzugreifen.
	 Es gibt auch F�lle, in denen die Elemente f�r die Webseiten auf der Serverseite verschleiert und unsichtbar sind, was es f�r den Webscraper schwieriger macht, sie zu lesen und fehleranf�llig ist.
	 Die Menge der verwertbaren Stammdaten hat sich aufgrund der begrenzten Komplexit�t des Webscrapers reduziert.
	 Dies ist auf Probleme bei der manuellen Dateneingabe sowie auf http/https-Probleme zur�ckzuf�hren (TODO: Erweitern, Anhang)
\end{itemize}


\subsection{Ressourcenplanung}
\label{sec:Ressourcenplanung}

\begin{itemize}
	\item In der �bersicht, die in Anhang A.2: Verwendete Ressourcen auf S. ii zu finden ist, sind alle f�r das Projekt verwendeten Ressourcen aufgef�hrt. 
	Dazu geh�ren Hard- und Software-Ressourcen sowie Personal. 
    Bei der Auswahl der verwendeten Software wurde darauf geachtet, dass diese kostenlos (z.B. als Open Source) zur Verf�gung steht oder das Unternehmen bereits Lizenzen daf�r besitzt. 
	Dadurch konnten die Projektkosten so gering wie m�glich gehalten werden.
\end{itemize}


\subsection{Entwicklungsprozess}
\label{sec:Entwicklungsprozess}
\begin{itemize}
	\item Bevor das Projekt umgesetzt wurde, musste ein bestimmter geeigneter Entwicklungsprozess gew�hlt werden. Dieser definiert die Prozedur, nach der die Umsetzung erfolgen soll.
	Im Verlauf des Projekts entschied sich der Autor f�r eine agile Methodik, das sogenannte Inkrementelle Vorgehensmodell. Bei der agilen
	Softwareentwicklung geht es darum, m�glichst schnell auf sich �ndernde Anforderungen reagieren zu k�nnen.
	
	\item Was ein inkrementelles Vorgehensmodell auszeichnet, ist, dass das zu entwickelnde System nicht im Voraus in allen Details genau geplant und dann in einem einzigen langen Durchgang entwickelt wird. 
	Stattdessen stehen Teile der Software bei neuen Erkenntnissen und Entdeckungen immer wieder neu im im Mittelpunkt.
	Die Entwicklung findet in kurzen Zeitspannen statt, nach denen jeweils ein neues Leistungsmerkmal erstellt wird, das dem Kunden gezeigt werden kann.
	Sollte der Kunde einen Anpassungswunsch haben oder eine neue Erkenntnis �ber die Identifikation von Anomalien gemacht werden,
	kann darauf bei der n�chsten Iteration schnell reagiert werden.
\end{itemize}
