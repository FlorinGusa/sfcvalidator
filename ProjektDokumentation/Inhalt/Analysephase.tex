% !TEX root = ../Projektdokumentation.tex
\section{Analysephase} 
\label{sec:Analysephase}


\subsection{Ist-Analyse} 
\label{sec:IstAnalyse}
\begin{itemize}
	\item Aufgrund der gro{\ss}en Anzahl von Kundenstamm Setzen ist eine manuelle Pr\"ufung der Daten nicht
	m\"oglich. Aufgrund vorgangenen Analysen geht man von Fehlerhaften Datens\"atzen im Umfang von 20\%. 
	Es bestehen Zweifel, wenn zwei Adressen vorhanden sind, da nicht klar ist, auf welche sie sich beziehen. 
	Die Bezeichnungen ''Stra{\ss}enadresse 1'' und ''Stra{\ss}enadresse 2'' sind interpretationsbed\"urftig und sollten nach Gelegenheit ge\"andert werden. 
	
\end{itemize}


\subsection{Wirtschaftlichkeitsanalyse}
\label{sec:Wirtschaftlichkeitsanalyse}
\begin{itemize}
	\item Das Vertriebsteam hat festgestellt, dass es regelm\"a{\ss}ig R\"uckl\"aufer f\"ur ung\"ultige E-Mails erh\"alt. 
	Au{\ss}erdem erhalten einige Interessenten und Kunden nicht den richtigen Inhalt, weil sie nicht richtig qualifiziert sind. 
	Man stellt fest, dass die E-Mail-Adresse das falsche Format hat und der Kundentyp falsch ist.
	Derzeit ist es erforderlich, dass ein Vertriebsmitarbeiter direkt vom Kunden \"uber \"Anderungen in den Kundenkontaktdaten informiert wird, oder dass der Kunde im Rahmen der Kommunikation darauf aufmerksam gemacht wird.
	Dies f\"uhrt jedoch zu Verz\"ogerungen, Unannehmlichkeiten und Missverst\"andnissen f\"ur die Kunden, was wiederum den wirtschaftlichen Faktor des Unternehmensimages beeintr\"achtigt.
	Die wirtschaftliche Ber\"ucksichtigung und die Entscheidung, ob die Realisierung des Projekts gerechtfertigt ist, wird in den folgenden Abschnitten vorgenommen.
\end{itemize}


\subsubsection{\gqq{Make or Buy}-Entscheidung}
\label{sec:MakeOrBuyEntscheidung}
\begin{itemize}
	\item Da es sich bei den eingehenden Dokumenten um firmenspezifische Prozesse der Fanuc Europe GmbH handelt, ist eine L\"osung durch ein zugekauftes Produkt nicht m\"oglich. 
	Daher muss eine L\"osung durch die Fanuc Europe GmbH entwickelt werden.
\end{itemize}


\subsubsection{Projektkosten}
\label{sec:Projektkosten}
\begin{itemize}
	\item Die Projektkosten, die durch die Erstellung dieser Anwendung entstanden sind, werden im Folgenden beschrieben.
	Es ist zu beachten, dass die Zahlen aus Datenschutzgründen angepasst wurden und daher nicht korrekt sind.
	Darin enthalten sind auch fixe Kosten wie Strom, Miete und Lizenzen, aber auch Zeitvergütungen für Mitarbeiter, die diese Anwendung bei der Erstellung und auch bei der Dokumentation unterstützt haben.
\end{itemize}

\paragraph
Der Auszubildende erhält einen Stundensatz von 15 \eur , während das reguläre Personal mit 35 \eur entlohnt wird. Hinzu kommen Kosten in die Höhe von 10 \eur pro Stunde für die Nutzung der Ressourcen.
\tabelle{Kostenaufstellung}{tab:Kostenaufstellung}{Kostenaufstellung.tex}


\subsubsection{Amortisationsdauer}
\label{sec:Amortisationsdauer}
\begin{itemize}
	\item Im Folgenden wird festgestellt, an welchem Punkt sich die Entwicklung der Software amortisiert hat.
    Durch die Implementierung dieser Software in die internen Gesch\"aftsabl\"aufe wird der Faktor Datenqualit\"at erh\"oht. 
 	Es gibt eine gro{\ss}e Anzahl verschiedener Dimensionen und Kriterien, mit denen die Datenqualit\"at beschrieben werden kann. 
	Durch die Art und Weise, wie die Software aufgebaut ist, wird das Potenzial f\"ur Unannehmlichkeiten f\"ur den Kunden reduziert - was den potenziellen Handel schneller und konsistenter macht. Die CRM-Administratoren haben au{\ss}erdem einen besseren \"Uberblick und k\"onnen inkonsistente Daten besser identifizieren.
	Da es m\"oglich ist, die Software automatisch im Rahmen von geplanten Arbeitsabl\"aufen auszuf\"uhren, ist der manuellen Arbeitsaufwand sehr gering, wodurch die Personalkosten auf ein Minimum reduziert werden.
	Im Folgenden soll nun die Arbeitszeitersparnis in tabellarischer Form ermittelt werden. Die Anzahl der Vorg\"ange
	pro Quartal und die Zeit pro Vorgang wurden durch den Salesforce-Administrator ermittelt
\end{itemize}

\paragraph{Beispielrechnung (verk\"urzt)}
Bei einer Zeiteinsparung von 10 Minuten am Tag für jeden der 5 Salesforce-Administroren und 220 Arbeitstagen im Jahr ergibt sich eine gesamte Zeiteinsparung von 

TODO: Rechnen

Die Amortisationszeit betr�gt also...


\paragraph{Beispiel}
%Ein Beispiel f�r eine Entscheidungsmatrix findet sich in Kapitel~\ref{sec:Architekturdesign}: \nameref{sec:Architekturdesign}.


\subsection{Anwendungsf\"alle}
\label{sec:Anwendungsfaelle}
\begin{itemize}
\item Wichtige Anwendungsf\"alle f\"ur dieses Projekt sind Verfahren, die in den t\"aglichen Arbeitsablauf eines Salesforce-Administrators einflie{\ss}en. Ein Salesforce-Admin kann planen, wann das Programm ausgef\"uhrt wird, ohne dass danach viel Input n\"otig ist. 

Eine grafische Darstellung des Anwendungsfalls ist in Abbildung 2 zu finden. 

Voraussetzung daf\"ur ist, dass der Salesforce-Administrator Zugang hat und die richtigen Daten zur Analyse ausw\"ahlt. Hierf\"ur wird eine Benutzerdokumentation erstellt, die in Anhang 3 zu finden ist.

Diese Anwendung ist nicht nur f\"ur die Korrektur alter Kundendaten vorgesehen, sondern kann auch f\"ur jeden neuen Datensatz, der w\"ochentlich in Salesforce eingegeben wird. Hierf\"ur wird w\"ochentlich ein Bericht mit allen neuen Kontodaten erstellt und in das Programm \"ubertragen.
Das wichtigste Erfolgsszenario besteht darin, dass das Programm aufschlussreiche Daten liefert, die keine gro{\ss}en Eingaben seitens des Benutzers erfordern.
F\"ur den Fall, dass der Webscraper die falschen Daten abruft oder nicht in der Lage ist, die vorhandenen Daten zu verarbeiten, werden Fehlerprotokolle exportiert, um dem Benutzer eine Vorstellung davon zu geben, warum die Daten nicht ausgewertet werden k\"onnen
\end{itemize}

%\paragraph{Beispiel}
%Ein Beispiel f�r ein Use Case-Diagramm findet sich im \Anhang{app:UseCase}.


\subsection{Lastenheft}
\label{sec:Lastenheft}
\begin{itemize}
\item Bevor das Projekt in die Umsetzungsphase eintritt, wurde ein Lastheft erstellt, in dem alle von den Salesforce-Administratoren gestellten Anfragen aufgelistet sind. Die Anwendung muss alle aufgef\"uhrten Aufgaben erf\"ullen.
Alle Einzelkriterien wurden in User Stories aufgeteilt, die in einem Kanban-Board mit Trello organisiert wurden. Ein Abschnitt ist auf Anlage 4 zu sehen.
\end{itemize}

