
\subsection{Lastenheft (Auszug)}
\label{app:Lastenheft}
Es folgt ein Auszug aus dem Lastenheft mit Fokus auf die Anforderungen:

Die Anwendung muss folgende Anforderungen erf\"ullen: 
\begin{enumerate}[itemsep=0em,partopsep=0em,parsep=0em,topsep=0em]
\item Herauslesen der Stammdaten
	\begin{enumerate}
	\item Die Anwendung muss Kundendaten entweder als CSV oder direkt aus Salesforce verarbeiten k\"onnen.
	\item Die Anwendung muss die Daten auf einheitliche Weise formatieren und Formatierungsregeln anwenden.
	\end{enumerate}
\item Darstellung der Daten
	\begin{enumerate}
	\item Die Anwendung muss eine einfache M\"oglichkeit bieten, nach Datensegmenten zu suchen und Berichte auf der Grundlage dieser Segmente zu erstellen.
	\item Fehler und Anomalien m\"ussen dem Nutzer deutlich angezeigt werden.
	\item Der Benutzer sollte die M\"oglichkeit haben, geplante Jobs zu erstellen, zu sehen und zu l\"oschen
	\item Die Anwendung sollte die Daten in Salesforce nicht \"uberschreiben, sondern den Benutzer auf Diskrepanzen aufmerksam machen
	\item CSV-Daten, die direkt vom Dataloader kommen, sollten ohne Probleme in der Anwendung eingelesen werden k\"onnen 
 	\item Alle eingelesenen Daten m\"ussen in der Benutzeroberfl\"ache sichtbar sein
	\end{enumerate}
\item Webscraper
	\begin{enumerate}
	\item Der Webscraper muss in der Lage sein, eine gegebene URL zu validieren und zu pr\"ufen, ob die entsprechende Webseite verf\"ugbar ist.
	\item Der Webscraper sollte nicht versuchen, ung\"ultig formatierte URLs anzupassen.
	\item Der Benutzer sollte die M\"oglichkeit haben, Einstellungen f\"ur den Webscraper vorzunehmen, einschliesslich der Anzahl der maximalen Seiten und ob http-Seiten (unsichere Seiten) ausgelesen werden d\"urfen
	\item Der Webscraper muss auf der Kommandozeile laufen k\"onnen, unabh\"angig von der Schnittstelle.
    \item Wenn m\"oglich, soll der Webscraper versuchen nach einer Kontaktseite zu suchen, wenn die Informationen nicht auf der Homepage aufgef\"uhrt sind.
\end{enumerate}
	
\item Sonstige Anforderungen
	\begin{enumerate}
	\item Die Anwendung sollte den Salesforce-Administratoren auf einem gemeinsamen Dateiserver zur Verf\"ugung gestellt werden
	\item Die Daten der Anwendung m\"ussen jede Nacht \bzw nach jedem \acs{SVN}-Commit automatisch aktualisiert werden. 
	\item Die Anwendung soll jederzeit erreichbar sein.
	\item Da sich die Entwickler auf die Anwendung verlassen, muss diese korrekte Daten liefern und darf keinen Interpretationsspielraum lassen.
 	\item Mit dem Fortschreiten des CRM-Projekts m\"ussen in sp\"ateren Phasen neue Datentypen implementiert werden.
	\end{enumerate}
\end{enumerate}
