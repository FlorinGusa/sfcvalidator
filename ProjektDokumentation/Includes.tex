
%\pagestyle{empty}		% keine Kopf und Fu�zeile (k. Seitenzahl)
%\pagestyle{headings}	% lebender Kolumnentitel  

%% Deutsche Anpassungen %%%%%%%%%%%%%%%%%%%%%%%%%%%%%%%%%%%%%
\usepackage[ngerman]{babel}
\usepackage[ansinew]{inputenc}
\pdfminorversion=5 
\usepackage[dvips,final]{graphicx} % Einbinden von JPG-Grafiken erm�glichen
\usepackage{graphics} % keepaspectratio
\usepackage{floatflt} % zum Umflie�en von Bildern
\graphicspath{{Bilder/}} % hier liegen die Bilder des Dokuments


% PDF-Optionen -----------------------------------------------------------------
\usepackage{pdfpages}
\pdfminorversion=5 % erlaubt das Einf�gen von pdf-Dateien bis Version 1.7, ohne eine Fehlermeldung zu werfen (keine Garantie f�r fehlerfreies Einbetten!)
\usepackage[
    bookmarks,
    bookmarksnumbered,
    bookmarksopen=true,
    bookmarksopenlevel=1,
    colorlinks=true,
% diese Farbdefinitionen zeichnen Links im PDF farblich aus
    anchorcolor=AOBlau,% Ankertext
    citecolor=AOBlau, % Verweise auf Literaturverzeichniseintr�ge im Text
    filecolor=AOBlau, % Verkn�pfungen, die lokale Dateien �ffnen
    menucolor=AOBlau, % Acrobat-Men�punkte
    urlcolor=AOBlau,
% diese Farbdefinitionen sollten f�r den Druck verwendet werden (alles schwarz)
    %linkcolor=black, % einfache interne Verkn�pfungen
    %anchorcolor=black, % Ankertext
    %citecolor=black, % Verweise auf Literaturverzeichniseintr�ge im Text
    %filecolor=black, % Verkn�pfungen, die lokale Dateien �ffnen
    %menucolor=black, % Acrobat-Men�punkte
    %urlcolor=black,
%
    %backref, % Quellen werden zur�ck auf ihre Zitate verlinkt
    pdftex,
    plainpages=false, % zur korrekten Erstellung der Bookmarks
    pdfpagelabels=true, % zur korrekten Erstellung der Bookmarks
    hypertexnames=false, % zur korrekten Erstellung der Bookmarks
    linkcolor=black,
    linktoc=all,
]{hyperref}