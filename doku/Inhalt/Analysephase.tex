% !TEX root = ../Projektdokumentation.tex
\section{Analysephase} 
\label{sec:Analysephase}


\subsection{Ist-Analyse} 
\label{sec:IstAnalyse}
\begin{itemize}
	\item Aufgrund der großen Anzahl von Kundenstamm Setzen ist eine manuelle Prüfung der Daten nicht
	möglich. Aufgrund vorgangenen Analysen geht man von Fehlerhaften Datensätzen im Umfang von 20%. 
	Es bestehen Zweifel, wenn zwei Adressen vorhanden sind, da nicht klar ist, auf welche sie sich beziehen. 
	Die Bezeichnungen "Straßenadresse 1" und "Straßenadresse 2" sind interpretationsbedürftig und sollten nach Gelegenheit geändert werden. 
	
\end{itemize}


\subsection{Wirtschaftlichkeitsanalyse}
\label{sec:Wirtschaftlichkeitsanalyse}
\begin{itemize}
	\item Das Vertriebsteam hat festgestellt, dass es regelmäßig Rückläufer für ungültige E-Mails erhält. 
	Außerdem erhalten einige Interessenten und Kunden nicht den richtigen Inhalt, weil sie nicht richtig qualifiziert sind. 
	Man stellt fest, dass die E-Mail-Adresse das falsche Format hat und der Kundentyp falsch ist.
	Derzeit ist es erforderlich, dass ein Vertriebsmitarbeiter direkt vom Kunden über Änderungen in den Kundenkontaktdaten informiert wird, oder dass der Kunde im Rahmen der Kommunikation darauf aufmerksam gemacht wird.
	Dies führt jedoch zu Verzögerungen, Unannehmlichkeiten und Missverständnissen für die Kunden, was wiederum den wirtschaftlichen Faktor des Unternehmensimages beeinträchtigt.
	Die wirtschaftliche Berücksichtigung und die Entscheidung, ob die Realisierung des Projekts gerechtfertigt ist, wird in den folgenden Abschnitten vorgenommen.

\end{itemize}


\subsubsection{\gqq{Make or Buy}-Entscheidung}
\label{sec:MakeOrBuyEntscheidung}
\begin{itemize}
	\item Da es sich bei den eingehenden Dokumenten um firmenspezifische Prozesse der Fanuc Europe GmbH handelt, ist eine Lösung durch ein zugekauftes Produkt nicht möglich. 
	Daher muss eine Lösung durch die Fanuc Europe GmbH entwickelt werden.
\end{itemize}


\subsubsection{Projektkosten}
\label{sec:Projektkosten}
\begin{itemize}
	\item Welche Kosten fallen bei der Umsetzung des Projekts im Detail an (\zB Entwicklung, Einführung/Schulung, Wartung)?
\end{itemize}

\paragraph{Beispielrechnung (verkürzt)}
Die Kosten für die Durchführung des Projekts setzen sich sowohl aus Personal-, als auch aus Ressourcenkosten zusammen.
Laut Tarifvertrag verdient ein Auszubildender im dritten Lehrjahr pro Monat \eur{1000} Brutto. 

\begin{eqnarray}
8 \mbox{ h/Tag} \cdot 220 \mbox{ Tage/Jahr} = 1760 \mbox{ h/Jahr}\\
\eur{1000}\mbox{/Monat} \cdot 13,3 \mbox{ Monate/Jahr} = \eur{13300} \mbox{/Jahr}\\
\frac{\eur{13300} \mbox{/Jahr}}{1760 \mbox{ h/Jahr}} \approx \eur{7,56}\mbox{/h}
\end{eqnarray}

Es ergibt sich also ein Stundenlohn von \eur{7,56}. 
Die Durchführungszeit des Projekts beträgt 70 Stunden. Für die Nutzung von Ressourcen\footnote{Räumlichkeiten, Arbeitsplatzrechner etc.} wird 
ein pauschaler Stundensatz von \eur{15} angenommen. Für die anderen Mitarbeiter wird pauschal ein Stundenlohn von \eur{25} angenommen. 
Eine Aufstellung der Kosten befindet sich in Tabelle~\ref{tab:Kostenaufstellung} und sie betragen insgesamt \eur{2739,20}.
\tabelle{Kostenaufstellung}{tab:Kostenaufstellung}{Kostenaufstellung.tex}


\subsubsection{Amortisationsdauer}
\label{sec:Amortisationsdauer}
\begin{itemize}
	\item Im Folgenden wird festgestellt, an welchem Punkt sich die Entwicklung der Software amortisiert hat.
    Durch die Implementierung dieser Software in die internen Geschäftsabläufe wird der Faktor Datenqualität erhöht. 
 	Es gibt eine große Anzahl verschiedener Dimensionen und Kriterien, mit denen die Datenqualität beschrieben werden kann. 
	Es gibt keine Richtlinien, wie viele Dimensionen man verwenden sollte. 
	DAMA zum Beispiel definiert sechzig Dimensionen, während die meisten DQM-Software-Anbieter in der Regel nur fünf Dimensionen vorschlagen.
	Durch die Art und Weise, wie die Software aufgebaut ist, wird das Potenzial für Unannehmlichkeiten für den Kunden reduziert - was den potenziellen Handel schneller und konsistenter macht.
	Die CRM-Administratoren haben außerdem einen besseren Überblick und können inkonsistente Daten besser identifizieren.
	Da es möglich ist, die Software automatisch im Rahmen von geplanten Arbeitsabläufen auszuführen, ist der manuellen Arbeitsaufwand sehr gering, wodurch die Personalkosten auf ein Minimum reduziert werden.
	Im Folgenden soll nun die Arbeitszeitersparnis in tabellarischer Form ermittelt werden. Die Anzahl der Vorgänge
	pro Quartal und die Zeit pro Vorgang wurden durch den Salesforce-Administrator ermittelt
\end{itemize}

\paragraph{Beispielrechnung (verkürzt)}
Bei einer Zeiteinsparung von 10 Minuten am Tag für jeden der 25 Anwender und 220 Arbeitstagen im Jahr ergibt sich eine gesamte Zeiteinsparung von 
\begin{eqnarray}
25 \cdot 220 \mbox{ Tage/Jahr} \cdot 10 \mbox{ min/Tag} = 55000 \mbox{ min/Jahr} \approx 917 \mbox{ h/Jahr} 
\end{eqnarray}

Dadurch ergibt sich eine jährliche Einsparung von 
\begin{eqnarray}
917 \mbox{h} \cdot \eur{(25 + 15)}{\mbox{/h}} = \eur{36680}
\end{eqnarray}

Die Amortisationszeit beträgt also $\frac{\eur{2739,20}}{\eur{36680}\mbox{/Jahr}} \approx 0,07 \mbox{ Jahre} \approx 4 \mbox{ Wochen}$.


\subsection{Nutzwertanalyse}
\label{sec:Nutzwertanalyse}
\begin{itemize}
	\item TODO: Kriterien gewichtung, Tabelle erstellen.
\end{itemize}

\paragraph{Beispiel}
Ein Beispiel für eine Entscheidungsmatrix findet sich in Kapitel~\ref{sec:Architekturdesign}: \nameref{sec:Architekturdesign}.


\subsection{Anwendungsfälle}
\label{sec:Anwendungsfaelle}
\begin{itemize}
	\item Welche Anwendungsfälle soll das Projekt abdecken?
	\item Einer oder mehrere interessante (!) Anwendungsfälle könnten exemplarisch durch ein Aktivitätsdiagramm oder eine \ac{EPK} detailliert beschrieben werden. 
\end{itemize}

\paragraph{Beispiel}
Ein Beispiel für ein Use Case-Diagramm findet sich im \Anhang{app:UseCase}.


\subsection{Qualitätsanforderungen}
\label{sec:Qualitaetsanforderungen}
\begin{itemize}
	\item Welche Qualitätsanforderungen werden an die Anwendung gestellt (\zB hinsichtlich Performance, Usability, Effizienz \etc (siehe \citet{ISO9126}))?
\end{itemize}


\subsection{Lastenheft/Fachkonzept}
\label{sec:Lastenheft}
\begin{itemize}
	\item Auszüge aus dem Lastenheft/Fachkonzept, wenn es im Rahmen des Projekts erstellt wurde.
	\item Mögliche Inhalte: Funktionen des Programms (Muss/Soll/Wunsch), User Stories, Benutzerrollen
\end{itemize}

\paragraph{Beispiel}
Ein Beispiel für ein Lastenheft findet sich im \Anhang{app:Lastenheft}. 
