% !TEX root = ../Projektdokumentation.tex
\section{Projektplanung} 
\label{sec:Projektplanung}
\label In der Projektplanung soll die notwendige Zeit und die benötigten Ressourcen sowie ein Ablauf der
Durchführung des Projektes geplant werden.


\subsection{Projektphasen}
\label{sec:Projektphasen}

\begin{itemize}
	\item Für die Umsetzung des Projekts standen dem Autor 70 Stunden zur Verfügung. Diese wurden vor dem Start des Projektes auf verschiedene Phasen verteilt, die während der Softwareentwicklung durchlaufen werden.
	Eine grobe Zeitplanung sowie die Hauptphasen können der Tabelle 1 entnommen werden: Grobzeitplanung
	entnommen werden. Dazu lassen sich die einzelnen Hauptphasen noch in kleinere Unterphasen untergliedern.
\end{itemize}

\paragraph{Beispiel}
Tabelle~\ref{tab:Zeitplanung} zeigt ein Beispiel für eine grobe Zeitplanung.
\tabelle{Zeitplanung}{tab:Zeitplanung}{ZeitplanungKurz}\\
Eine detaillierte Übersicht über diese Phasen befindet sich im \Anhang{app:Zeitplanung} A.1: Detaillierte Zeitplanung am
S. i.

\subsection{Abweichungen vom Projektantrag}
\label{sec:AbweichungenProjektantrag}

\begin{itemize}
	\item Im Projektantrag wurde festgelegt, dass bestimmte Bibliotheken mit Python für das Webscraping-System zusammen mit einem einfachen Codebeispiel verwendet werden.
	 Dies wurde jedoch geändert, da es mehrere dynamische Websites gibt, was bedeutet, dass der Benutzer zuerst mit der Website interagieren muss, um auf sie zuzugreifen.
	 Es gibt auch Fälle, in denen die Elemente für die Webseiten auf der Serverseite verschleiert und unsichtbar sind, was es für den Webscraper schwieriger macht, sie zu lesen und fehleranfällig ist.
	 \item TODO: Code Beispiele
\end{itemize}


\subsection{Ressourcenplanung}
\label{sec:Ressourcenplanung}

\begin{itemize}
	\item In der Übersicht, die in Anhang A.2: Verwendete Ressourcen auf S. ii zu finden ist, sind alle für das Projekt verwendeten Ressourcen aufgeführt. 
	Dazu gehören Hard- und Software-Ressourcen sowie Personal. 
    Bei der Auswahl der verwendeten Software wurde darauf geachtet, dass diese kostenlos (z.B. als Open Source) zur Verfügung steht oder das Unternehmen bereits Lizenzen dafür besitzt. 
	Dadurch konnten die Projektkosten so gering wie möglich gehalten werden.
\end{itemize}


\subsection{Entwicklungsprozess}
\label{sec:Entwicklungsprozess}
\begin{itemize}
	\item Bevor das Projekt umgesetzt wurde, musste ein bestimmter geeigneter Entwicklungsprozess gewählt werden. Dieser definiert die Prozedur, nach der die Umsetzung erfolgen soll.
	Im Verlauf des Projekts entschied sich der Autor für eine agile Methodik, das sogenannte Inkrementelle Vorgehensmodell. Bei der agilen
	Softwareentwicklung geht es darum, möglichst schnell auf sich ändernde Anforderungen reagieren zu können.
	
	\item Was ein inkrementelles Vorgehensmodell auszeichnet, ist, dass das zu entwickelnde System nicht im Voraus in allen Details genau geplant und dann in einem einzigen langen Durchgang entwickelt wird. 
	Stattdessen stehen Teile der Software bei neuen Erkenntnissen und Entdeckungen immer wieder neu im im Mittelpunkt.
	Die Entwicklung findet in kurzen Zeitspannen statt, nach denen jeweils ein neues Leistungsmerkmal erstellt wird, das dem Kunden gezeigt werden kann.
	Sollte der Kunde einen Anpassungswunsch haben oder eine neue Erkenntnis über die Identifikation von Anomalien gemacht werden,
	kann darauf bei der nächsten Iteration schnell reagiert werden.
\end{itemize}
